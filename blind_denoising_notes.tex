%with $X_i$ sampled as the output of a P-Net:
%$$
%X_ i = \mu_{\theta_p}(\Omega_{Y_i},\overline \Omega_{Y_i}) + \sigma_{\theta_p}(\Omega_{Y_i},\overline \Omega_{Y_i})\varepsilon_i
%$$
%and $\varepsilon_i$ a standard Gaussian random variable.


%In the following, let $X$ be the hidden signal at a given pixel, $Y$ be the corresponding observation corrupted with noise and $\Omega_Y$ be the noisy observations of the surrounding pixels in a given neighborhood of the pixel. Following \cite{laine2019high}, assume that conditionally on $\Omega_Y$, $X$ has a Gaussian distribution with mean $\mu_{\theta_p}$ and variance $\sigma_{\theta_p}^2$. Then, conditionally on $(\Omega_Y,X)$, assume that the noise has a Gaussian distribution with mean $0$ and variance $\sigma_{\theta_n}^2(X)$. In this case, the probability density of $Y$ conditionally on $\Omega_Y$ is
%$$
%p_{\theta}(y|\Omega_y) = \int \varphi_{\mu_{\theta_p},\sigma_{\theta_p}^2}(u)\varphi_{0,\sigma_{\theta_n}^2(u)}(y-u)\mathrm{d}u\,,
%$$
%where $\varphi_{\mu,\sigma^2}$ is the Gaussian probability density function with mean $\mu$ and variance $\sigma^2$.
%%To obtain this density, as $p(x,\varepsilon|\Omega_y) =p(x|\Omega_y)p(\varepsilon|\Omega_y,x)$, write for any function $h$
%%\begin{align*}
%% \mathbb{E}[h(Y)|\Omega_y] =   \mathbb{E}[h(X+\varepsilon)|\Omega_y] &= \int h(x+u)  \varphi_{\mu_p,\sigma_p^2}(x)\varphi_{0,\sigma_n^2(x)}(u)\mathrm{d}u\mathrm{d}x\,,\\
%% &= \int \varphi_{\mu_p,\sigma_p^2}(x) \left(\int h(x+u)  \varphi_{0,\sigma_n^2(x)}(u)\mathrm{d}u\right)\mathrm{d}x\,,\\
%% &= \int \varphi_{\mu_p,\sigma_p^2}(x) \left(\int h(y)  \varphi_{0,\sigma_n^2(x)}(y-x)\mathrm{d}y\right)\mathrm{d}x\,,\\
%%  &= \int h(y)\left(\int \varphi_{\mu_p,\sigma_p^2}(x)   \varphi_{0,\sigma_n^2(x)}(y-x)\mathrm{d}x\right)\mathrm{d}y\,.
%%\end{align*}
%As $\sigma_{\theta_n}^2$ depends on $u$ this integral cannot be computed explicitly. This model depends on the unknown parameter $\theta = (\theta_p,\theta_n)$. In the following, $(\mu_{\theta_p},\sigma^2_{\theta_p})$ and $\sigma^2_{\theta_n}$ are computed using deep neural networks and $\theta_p$ and $\theta_n$ are the weights and biases of these networks.
%
%The approach proposed in this paper is decomposed into two steps.
%\begin{enumerate}
%\item First, $\mu_{\theta_p}$ and $\sigma_{\theta_p}^2$  are obtained with the P-Net fed with $\Omega_y$ and $\overline \Omega_y$ at central position, $\sigma_{\theta_n}^2$ is  obtained with the N-Net fed with $x$.
%\item After a training procedure to estimate the parameters of these networks, the estimator of the unknow pixel given the observations $(y,\Omega_y)$ is set to $\mu_{\theta_p}(\Omega_y,y)$ as an approximation of the posterior mean.
%\end{enumerate}
%The loss function is defined by approximating the loglikelihood of the observations using samples from the P-net:
%$$
%\ell_{\theta}: y \mapsto \frac{1}{N}\sum_{i=1}^N \ell_{\theta}(X_i,Y_i|\Omega_{Y_i})\,,
%$$
%where
%\begin{multline*}
%\ell_{\theta}(X_i,Y_i|\Omega_{Y_i}) = \log(\sigma_{\theta_n}(X_i)^2) + \frac{(Y_i-X_i)^2}{\sigma_{\theta_n}(X_i)^2} \,,
%\end{multline*}
%with $X_i$ sampled as the output of a P-Net:
%$$
%X_ i = \mu_{\theta_p}(\Omega_{Y_i},\overline \Omega_{Y_i}) + \sigma_{\theta_p}(\Omega_{Y_i},\overline \Omega_{Y_i})\varepsilon_i
%$$
%and $\varepsilon_i$ a standard Gaussian random variable.
%In this paper, a Monte Carlo approximation of this likelihood is used to train the model.
%\begin{enumerate}
%    \item Compute $\nabla_u\sigma_n^2(u)$ and replace $\sigma_n^2$ by its first order Taylor expansion to obtain a linear function and then compute the integral. \textcolor{red}{More complex than expected...}
%    \item Replace the integral by a Monte Carlo estimate.
%\end{enumerate}

\section{Alternatives for baselines}
\subsection{Training with an approximation of the score}
An appealing training algorithm aims at approximating directly the score function $\theta\mapsto \nabla_{\theta} \log p_{\theta}(y|\Omega_y)$. Write
\begin{align*}
\log p_{\theta}(y|\Omega_y) &= \mathbb{E}[\log p_{\theta}(y|\Omega_y)|y,\Omega_y]\,,\\
&= \mathbb{E}\left[\log \frac{p_{\theta}(X,y|\Omega_y)}{p_{\theta}(X|y,\Omega_y)}\middle |y,\Omega_y\right]\;
\end{align*}
%and
%\begin{align*}
%\nabla_{\theta}\log p(y|\Omega_y) = \mathbb{E}\left[\nabla_{\theta}\log p(X,y|\Omega_y)\middle |y,\Omega_y\right] - \mathbb{E}\left[\nabla_{\theta}\log p(X|y,\Omega_y)\middle |y,\Omega_y\right]\,.
%\end{align*}
Using that
\begin{align*}
\mathbb{E}&\left[\nabla_{\theta}\log p_{\theta}(X|y,\Omega_y)\middle |y,\Omega_y\right] \\
&\hspace{1.5cm}= \int \frac{ \nabla_{\theta} p_{\theta}(x|y,\Omega_y)}{p_{\theta}(x|y,\Omega_y)}p_{\theta}(x|y,\Omega_y)\mathrm{d}x\,,\\
&\hspace{1.5cm} = \nabla_{\theta} \int  p_{\theta}(x|y,\Omega_y)\mathrm{d}x = 0\,,
\end{align*}
yields
$$
\nabla_{\theta}\log p_{\theta}(y|\Omega_y) = \mathbb{E}\left[\nabla_{\theta}\log p_{\theta}(X,y|\Omega_y)\middle |y,\Omega_y\right]\,.
$$
The first option is then to sample $(X_i)_{1\leqslant i \leqslant N}$ with distribution $p(x|y,\Omega_y)$ and approximate $\nabla_{\theta}\log p(y|\Omega_y)$ by
$$
\frac{1}{N}\sum_{i=1}^N \nabla_{\theta}\log p_{\theta}(X_i,y|\Omega_y)
$$
which boils down to choosing the cost function
$$
\ell_{\theta}: y \mapsto \frac{1}{N}\sum_{i=1}^N \ell_{\theta}(X_i,y|\Omega_y)\,,
$$
where
\begin{multline*}
\ell_{\theta}(X_i,y|\Omega_y) = \frac{1}{2\sigma^2_{\theta_p}(\Omega_y)}(X_i-\mu_{\theta_p}(\Omega_y))^2 \\ \frac{1}{2}\log(\sigma^2_{\theta_p}(\Omega_y)) + \frac{1}{2}\log(\sigma_{\theta_n}(X_i)^2) + \frac{1}{2\sigma_{\theta_n}(X_i)^2}(y-X_i)^2 \,.
\end{multline*}
This approach which seems appealing suffers from a major drawback as it is usually cumbersome to produce samples from $p(x|y,\Omega_y)$ as $p(x|y,\Omega_y)\propto \varphi_{\mu_p,\sigma_p^2}(x)\varphi_{0,\sigma_n^2(x)}(y-x)$.

\subsubsection{$\varepsilon$-biased importance sampling}
This cost function may be hard to compute as sampling $(X_i)_{1\leqslant i \leqslant N}$ with distribution $p_\theta(x|y,\Omega_y)$ is challenging (see above). An alternative is to sample $(X_i)_{1\leqslant i \leqslant N}$ with distribution $q(x|\Omega_y,y)$ and associate an importance weight to each sample. As
\begin{align*}
p_\theta(x|\Omega_y,y) &\propto p_\theta(x,\Omega_y,y) \propto p_{\theta_p}(x|\Omega_y)p_{\theta_n}(y|x,\Omega_y)\,\\
&\propto \varphi_{\mu_p(\Omega_y),\sigma^2_p(\Omega_y)}(x)   \varphi_{0,\sigma_n^2(x)}(y-x)\,,
\end{align*}
if $X_i \sim q(\cdot|\Omega_y,y)$, then write $\omega_i = p_\theta(X_i|\Omega_y,y)/q(X_i|\Omega_y,y)$ and consider the cost function
$$
y \mapsto \omega_i \ell_{\theta}(X_i,y|\Omega_y) \,,
$$
with a stop gradient on the $\omega_i$ and on the $X_i$ as the gradient must only affect $\ell_{\theta}(X_i,y|\Omega_y)$. The explicit expression of $\omega_i$ is not available since
$$
\omega_i = \frac{\varphi_{\mu_p(\Omega_y),\sigma^2_p(\Omega_y)}(X_i)   \varphi_{0,\sigma_n^2(X_i)}(y-X_i)}{q(X_i|\Omega_y,y) \int \varphi_{\mu_p(\Omega_y),\sigma^2_p(\Omega_y)}(x)   \varphi_{0,\sigma_n^2(x)}(y-x) \mathrm{d} x}
$$
\paragraph{Solution 1.} Assume that $\omega_i$ is close to one (and then set it to 1) and choose for $q(\cdot|\Omega_y,y)$ the P-Net without a mask or with the mask i.e. sample $X_i$ with the P-Net with or without the mask.

\paragraph{Solution 2.} Sample $X_i$ with the P-Net with or without the mask and compute $\omega_i$ by replacing the integral by a numerical approximations.

\paragraph{Solution 3.} Sample $X_i$ with the P-Net with (1) or without (2) the mask and compute the auto-normalized version of $\omega_i$.
\begin{enumerate}
\item $\omega_i = \varphi_{0,\sigma_n^2(X_i)}(y-X_i)$
\item $\omega_i = \frac{ \varphi_{\mu_p(\Omega_y),\sigma^2_p(\Omega_y)}(X_i)   \varphi_{0,\sigma_n^2(X_i)}(y-X_i)}{\varphi_{\mu_p(\Omega_y,y),\sigma^2_p(\Omega_y,y)}}$
\end{enumerate}

\subsubsection{Improved importance sampling}
The previous importance sampling estimator is not efficient as all samples are obtained without considering $y$ nor $\sigma_n^2$. An alternative is to sample $X_i$ from the proposal distribution:
$$
q(x|\Omega_y,y)\propto \varphi_{\mu_p,\sigma_p^2}(x)   \varphi_{0,\sigma_n^2(\mu_p)}(y-x)\,.
$$
The samples $X_i$ are then Gaussian with mean and variance
$$
\sigma^2(y) = \frac{\sigma_p^2\sigma_n^2(\mu_p)}{\sigma_n^2(\mu_p)+\sigma_p^2} \quad\mbox{and}\quad \mu(y) = \sigma^2(y)\left(\frac{\mu_p}{\sigma_p^2} + \frac{y}{\sigma_n^2(\mu_p)}\right)\,.$$
The importance weight associated with $X_i$ is $\omega_i = \varphi_{0,\sigma_n^2(X_i)}(y-X_i)/\varphi_{0,\sigma_n^2(\mu_p)}(y-X_i)$. Consider the cost function
$$
y \mapsto \sum_{i=1}^N\frac{\omega_i}{\sum_{j=1}^N\omega_j}  \log p(X_i,y|\Omega_y)\,,
$$
with a stop gradient on the $\omega_i$ and on the $X_i$ as the gradient must only affect $ \log p(X_i,y|\Omega_y)$.

%\paragraph{Training.} The first option seems to be the most promising as the only approximation is a linear approximation of the N-net producing $\sigma_n^2$ to compute the integral. The second option seems more closely related to what is being processed with the ongoing simulations. The simplest version of the second option is to write the following approximation of $p(y|\Omega_y)$:
%$$
%\hat p(y|\Omega_y) = \varphi_{0,\sigma_n^2(\mu_p + \sigma_p\zeta)}(y-\mu_p - \sigma_p\zeta)\,,
%$$
%with $\zeta \sim \mathcal{N}(0,1)$. In this estimator, $\sigma_n^2(\mu_p + \sigma_p\zeta)$ is the output of the N-Net when fed with $\mu_p + \sigma_p\zeta$, i.e. when fed with a sample from the distribution $p(x|\Omega_y)$. The training loss function is then
%$$
%y\mapsto \frac{1}{2}\log(\sigma_n^2(\mu_p + \sigma_p\zeta)) + \frac{1}{2\sigma_n^2(\mu_p + \sigma_p\zeta)}(y-\mu_p - \sigma_p\zeta)^2\,.
%$$

%\paragraph{Inference.} The inference process requires to draw samples from $p(x|\Omega_y,y)$ (or to compute its mean). Note that
%$$
%p(x|\Omega_y,y) \propto p(x,\Omega_y,y) \propto p(x|\Omega_y)p(y|x,\Omega_y)\propto \varphi_{\mu_p,\sigma_p^2}(x)   \varphi_{0,\sigma_n^2(x)}(y-x)\,.
%$$
%Assume that there exists $\varepsilon>0$ such that for all $x$, $\sigma_n^2(x)>\varepsilon$. Then,
%$$
%\varphi_{\mu_p,\sigma_p^2}(x)   \varphi_{0,\sigma_n^2(x)}(y-x)\leqslant (2\pi \varepsilon )^{-1/2}\varphi_{\mu_p,\sigma_p^2}(x)\,.
%$$
%An acceptance-rejection method may then be used to sample from $p(x|\Omega_y,y)$. To obtain one sample the algorithm proceeds as follows. Sample $X$ with Gaussian distribution with mean $\mu_p$ and variance $\sigma_p^2$ and U uniform in $(0,1)$. If $U \leqslant (2\pi \varepsilon )^{1/2}\varphi_{0,\sigma_n^2(X)}(y-X)$ then accept $X$. Otherwise sample again $(X,U)$ until acceptance.





%\bigskip

%\textcolor{red}{The following will not be used... I guess}.

%\bigskip

%\textcolor{red}{En cours (la c'est une version gaussienne multivariee), on devrait pouvoir en tirer une version portant uniquement sur la loi du pixel manquant, plus proche de ce qui est deja implemente...}
%For any pixel $X_i$, let $Y_{\mathcal{V}_i}$ be the corrupted signal in a neighborhood of $X_i$. In the following $X_{-i}$ stands for the pixel values in the neighborhood except $X_i$.
%The posterior distribution of the signal in a given position $i$ is therefore given by
%$$
%p(X_i|Y) = p(X_i|Y_{\mathcal{V}_i}) = \frac{p(X_i;Y_{\mathcal{V}_i})}{p(Y_{\mathcal{V}_i})} \propto p(X_i;Y_{\mathcal{V}_i}) =p(X_i) p(Y_{i}|X_i)\int p(x_{-i}|X_i)p(Y_{-i}|x_{-i})d x_{-i}
%$$
%Assume that the prior distribution of $X$ is a Gaussian distribution with mean $\mu$ and covariance matrix $\Sigma$. Then, conditionally on $X_i$, $X_{-i}$ has a Gaussian distribution with mean $\tilde \mu_i = \mu_{-i} + \Sigma_{-i;i}\sigma^{-2}_{i}(X_{i}-\mu_{i})$ and variance $\tilde \Sigma_i = \Sigma_{-i;-i} - \Sigma_{-i;i}\Sigma_{i;-i}\sigma_i^{-2}$.
%\begin{align*}
%\log p(X_i) &= \frac{1}{2}\log \sigma_i^2 - \frac{1}{2\sigma_i^2}(X_i-\mu_i)^2\eqsp,\\
%\log p(Y_i|X_i) &=  - \frac{1}{2}\log \sigma_n^2 - \frac{1}{2\sigma_n^2}(Y_i-X_i)^2\eqsp,\\
%\log \int p(x_{-i}|X_i)p(Y_{-i}|x_{-i})d x_{-i} &= - \frac{1}{2}\log \mathrm{det}(2\pi (\tilde \Sigma_i +\sigma_n^2I)) - \frac{1}{2}(Y_{-i}-\tilde \mu_i)^\top(\tilde \Sigma_i +\sigma_n^2I)^{-1}(Y_{-i}-\tilde \mu_i)\eqsp.
%\end{align*}
%Therefore
%\begin{multline*}
%\log p(X_i;Y_{\mathcal{V}_i}) = \frac{1}{2}\log \sigma_i^2 - \frac{1}{2\sigma_i^2}(X_i-\mu_i)^2 - \frac{1}{2}\log \sigma_n^2 - \frac{1}{2\sigma_n^2}(Y_i-X_i)^2\\
%- \frac{1}{2}\log \mathrm{det}(2\pi (\tilde \Sigma_i +\sigma_n^2I)) - \frac{1}{2}(Y_{-i}-\tilde \mu_i)^\top(\tilde \Sigma_i +\sigma_n^2I)^{-1}(Y_{-i}-\tilde \mu_i)\eqsp.
%\end{multline*}
%The EM algorithm updates iteratively the estimator $(\theta_p)_{p\geqslant 0}$ by maximizing
%$$
%Q(\theta;\theta_p) = \mathbb{E}_{\theta_p}[\log p(X_i;Y_{\mathcal{V}_i})|Y_{\mathcal{V}_i}]\eqsp.
%$$
%  An approximation of this quantity can be obtained by sampling $\zeta$ with the same law as the conditional distribution of $X_i$ given $Y_{\mathcal{V}_i}$ and replace this intermediate quantity by
%$$
%\widehat Q(\theta;\theta_p) = \log p(\zeta;Y_{\mathcal{V}_i})\eqsp.
%$$

%\subsection{Base model}
%\label{sec:basemodel}

%
%\begin{equation}
%    c(\vh, \vt)_i = \vyh_i = \frac{\langle \vh, \vt_i \rangle}{\Vert \vh \Vert_2 \Vert \vt_i \Vert_2}\,.
%    \label{eq:cosine-classifier}
%\end{equation}
